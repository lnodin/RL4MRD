\begin{abstract}
	Các thực thể trong thế giới này có thể được tổ chức thành một đồ thị mà quan hệ giữa những thực thể có những kiểu khác nhau này có thể là những cạnh với những kiểu khác nhau. Đó là đồ thị tri thức. Dữ liệu đồ thị tri thức không bao giờ có thể hoàn thiện. Hiện nay, có nhiều phương pháp đề xuất để mà cố gắng hoàn thiện hay khám phá những dữ liệu chưa được nhìn thấy (unseen facts) hay những mối quan hệ tiềm tàn (latent relatioships) bên trong loại dữ liệu này. Một trong những phương pháp tiếp cận hiện nay cho vấn đề này là suy luận đồ thị tri thức đa bước (multi-hop knowledge graph reasoning). Quá trình thực thi của phương pháp này có thể thể hiện như một bài toán quyết định xâu chuỗi (serilized decision problem), và có thể được giải quyết bằng phương pháp học tăng cường (reinforcement learning). Dựa trên công trình đã được công bố trong khoảng thời gian gần đây, RL-based multi-hop KG reasoning model Path Additional Action space Ranking (PAAR), trong bản technical report này, chúng tôi đề xuất mô hình cải tiến cho PAAR dựa trên nhúng đồ thị tri thức Fourier (Fourier-Knowledge Graph Embeddig) và tối ưu hóa cho quá trình học các bản nhúng hiệu quả hơn thông qua phương pháp Quasi-hyperbolic momentum và Adam. Để thêm vào những bản nhúng hữu ích hơn, các vector Fourier-KGE được thêm vào không gian trạng thái và giúp cho việc cải thiện tính thể hiện của không gian trạng thái. Tương tự như PAAR, chúng tôi giải quyết vấn đề thưa phần thưởng (reward sparsity problem) trong học tăng cường bằng cách sử dụng hàm tính điểm (score function) của Fourier-KGE như một phần thưởng mềm (soft-reward). Các kết quả thực nghiệm được báo cáo lại thành dạng bảng dựa trên các bộ dữ liệu thí nghiệm và tái thực nghiệm kết quả của bài báo gốc.
\end{abstract}
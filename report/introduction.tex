\section{Giới thiệu}
Hiện nay, máy học với dữ liệu đồ thị và các ứng dụng của nó nhận được nhiều sự chú ý từ lẫn cộng đồng nghiên cứu và doanh nghiệp. Lấy động lực từ sự tiềm năng của cấu trúc dữ liệu đồ thị như một ngôn ngữ phổ phát cho việc mô tả và phân tích nhiều cấu trúc dữ liệu thế giới thật như cấu trúc phân tử, cấu trúc khung xương con người, mạng máy tính, mạng thức ăn, mạng xã hội hay mạng lưới nơ-ron trong não bộ con người, đồ thị trở thành một công cụ hiệu quả cho nghiên cứu lẫn ứng dụng vào nhiều lĩnh vực khác nhau. Dữ liệu đồ thị đa quan hệ như đồ thị tri thức đóng vai trò quan trọng vì nó mô tả và biểu diễn những miền tri thức phức tạp. Khai thác những loại dữ liệu như thế làm tiền đề cho nhiều ứng dụng phía sau. Tuy nhiên, các vấn đề phải đối mặt với dữ liệu đồ thị rất nhiều và phức tạp, ví dụ như: độ phức tạp về không gian và thời gian của mô hình; hay vấn đề biểu diễn đồ thị (mạng nói chung) sao cho bảo toàn cấu trúc và những thông tin tương tác tiềm tàn bên trong nó.

Dữ liệu đồ thị tri thức thường được khởi tạo và làm giàu từ nhiều nguồn tài nguyên khác nhau (các nguồn tài nguyên phải đảm bảo tính hợp pháp) như ngữ liệu văn bản, các nguồn dữ liệu có cấu trúc như JSON, XML, hay CSV. Do những kỹ thuật thu thập, xử lý và khởi tạo cho đồ thị tri thức sơ khai thường dựa trên các cơ chế thủ công hoặc bán tự động, thế nên nó thường không hoàn thiện. Để giải quyết vấn đề này, một trong các cách tiếp cận hiện nay là suy diễn đồ thị tri thức (knowledge graph reasoning). Một trong nhiều các phương pháp cho suy diễn đồ thị tri thức là nhúng đồ thị tri thức (knowledge graph embedding). Phương pháp này xây dựng hàm ánh xạ những bộ ba dữ kiện $(h, r, t)$ vào không gian vector liên tục thấp chiều trong khi vẫn bảo toàn cấu trúc nội tại của những đối tượng bên trong đồ thị (Ở đây, đối tượng bao gồm những thực thể và quan hệ liên kết giữa chúng).

Xem xét bài toán với truy vấn $(?, \mathbf{r}, \mathbf{t})$, các mô hình nhúng đồ thị tính toán vector đặc trưng (representation vector) $\mathbf{v}$ từ những thông tin sẵn có. Quá trình dự đoán liên kết được thực thi bằng cách truy vẫn những thực thể mục tiêu $\mathbf{h}$ nào có vector đặc trưng "gần" với $\mathbf{v}$ nhất. Tuy nhiên, phương pháp có nhược điểm khi ngăn chặn tính kết hợp giữa các quan hệ. Để giải quyết vấn đề này, bài toán suy diễn đồ thị phát triển thành dạng suy diễn những bộ ba dữ kiện bị thiếu bằng những thông tin tổng hợp được từ các đường đi đa bước (multi-hop paths). Quá trình suy diễn đa bước này có thể được biểu diễn thành bài toán chuỗi quyết định (serialized decision problem) và hoàn toàn có thể giải quyết với các mô hình học tăng cường.

Lấy cảm hứng từ tác vụ xử lý ảnh trong miền tần số, phép biến đổi Fourier cho ta nhiều lợi ích trong việc tính toán và xử lý dữ liệu ảnh số. Một cách cụ thể, ta dễ dàng lọc những tần số không cần thiết (thông tin nhiễu) và thực hiện các xử lý trong miền tần số nhanh hơn rất nhiều so với miền không gian. Dựa trên công trình multi-hop KG reasoning PAAR (Path Additional Action-space Ranking), chúng tôi xây dựng và tối ưu mô hình trong miền tần số với mô hình nhúng đồ thị Fourier. Một cách cụ thể, trong bản technical report, chúng tôi đóng góp:
\begin{itemize}
    \item Đề xuất mô hình nhúng đồ thị trong miền tần số sử dụng phép biến đổi Fourier (Fourier-Knowledge Graph Embedding), tăng tốc độ xử lý và tính toán lọc bỏ nhiễu dựa trên miền không gian mới.
    \item Cải thiện tối ưu hóa mô hình so với phương pháp nhúng trên miền số phức bằng cách sử Quasi-Hyperbolic momentum và Adam.
    \item Thực hiện và báo cáo lại các kết quả trên các tập dữ liệu phòng thí nghiệm bao gồm FB15k-237 10\% và FB15k-237 20\%. Mã nguồn thực hiện đề tài được công khai ở Github\footnote{https://github.com/m32us/RL4MRD}
\end{itemize}